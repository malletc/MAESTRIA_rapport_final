\section{IMPACT DU PROJET}
\label{sec:impact}


\ANRinfo{Ce rapport rassemble des éléments nécessaires au bilan du projet et plus globalement permettant d’apprécier l’impact du programme à différents niveaux.}

\subsection{Indicateurs d’impact}
\subsubsection*{Nombre de publications et de communications (à détailler en E.2)}
\ANRinfo{Comptabiliser séparément les actions monopartenaires, impliquant un seul partenaire, et les actions multipartenaires  résultant d’un travail en commun.
Attention : éviter une inflation artificielle des publications, mentionner uniquement celles qui résultent directement du projet (postérieures à son démarrage, et qui citent le soutien de l’ANR et la référence du projet).}


\begin{table}[htbp]
\small
    \centering
    \begin{tabular}{|p{3.25cm}|p{3.25cm}|c|c|}
    \hline
&  & Publications & Publications\\
 & &multipartenaires & monopartenaires \\
\hline

\multirow{3}{*}{International}&Revues à comité de lecture&&\\\cline{2-4}
&Ouvrages ou chapitres d’ouvrage&&\\\cline{2-4}
&Communications (conférence)&&\\\hline
\multirow{3}{*}{National}&Revues à comité de lecture&&\\\cline{2-4}
&Ouvrages ou chapitres d’ouvrage&&\\\cline{2-4}
&Communications (conférence)&&\\\hline
\multirow{3}{*}{Actions de diffusion}&Articles vulgarisation&&\\\cline{2-4}
&Conférences vulgarisation&&\\\cline{2-4}
&Autres&&\\
\hline
    \end{tabular}
    \label{tab:publis}
\end{table}


\subsubsection*{Autres valorisations scientifiques (à détailler en E.3)} 
\ANRinfo{Ce tableau dénombre et liste les brevets nationaux et internationaux, licences, et autres éléments de propriété intellectuelle consécutifs au projet, du savoir faire, des retombées diverses en précisant les partenariats éventuels. Voir en particulier celles annoncées dans l’annexe technique). }

\begin{table}[htbp]
\small
    \centering
    \begin{tabular}{|p{5.5cm}|p{6cm}|}
    \hline
& Nombre, années et commentaires \\
&(valorisations avérées ou probables)\\\hline
Brevets internationaux obtenus& \\\hline
Brevet internationaux en cours d’obtention& \\\hline
Brevets nationaux obtenus& \\\hline
Brevet nationaux en cours d’obtention& \\\hline
Licences d’exploitation (obtention / cession)& \\\hline
Créations d’entreprises ou essaimage& \\\hline
Nouveaux projets collaboratifs & \\\hline
Colloques scientifiques& \\\hline
Autres (préciser)& XX dépôts de code Open Source + Participation des deux partenaires aux développements de la librairie $\iota^2$ et à son \href{https://framagit.org/iota2-project/iota2/-/wikis/Project-Steering-Committee}{comité de pilotage}. \\
\hline
    \end{tabular}
    \label{tab:valo_other}
\end{table}


\subsection{Liste des publications et communications}
\ANRinfo{Répertorier les publications résultant des travaux effectués dans le cadre du projet en suivant les normes éditoriales habituelles. En ce qui concerne les conférences, on spécifiera les conférences invitées.}

%% Nos publis ANR specifiquement dans ce bib
%% Merci de mettre le keyword ci-dessous pour que cela marche
\printbibliography[keyword={LASTIG-CESBIO-j},title={Journaux}]
\printbibliography[keyword={LASTIG-CESBIO-c},title={Conférences}]
\printbibliography[keyword={LASTIG-CESBIO-a},title={Autres communications}]

\subsection{Liste des éléments de valorisation}
\ANRinfo{La liste des éléments de valorisation inventorie les retombées (autres que les publications) décomptées dans le deuxième tableau de la section E.1. On détaillera notamment :
\begin{itemize}
\item brevets nationaux et internationaux, licences, et autres éléments de propriété intellectuelle consécutifs au projet.
\item logiciels et tout autre prototype
\item actions de normalisation 
\item lancement de produit ou service, nouveau projet, contrat,…
\item le développement d’un nouveau partenariat,
\item la création d’une plate-forme à la disposition d’une communauté
\item création d’entreprise, essaimage, levées de fonds
\item autres (ouverture internationale,..)
\end{itemize}
Elle en précise les partenariats éventuels. Dans le cas où des livrables ont été spécifiés dans l’annexe technique, on présentera ici un bilan de leur fourniture.}
Le projet s'est focalisé sur le renforcement des composantes méthodologiques de la chaîne logicielle libre $\iota^2$ (\href{https://framagit.org/iota2-project/iota2}{https://framagit.org/iota2-project/iota2}). Les codes développés ont suivi la licence de $\iota^2$ (GNU Affero General Public License v3.0). Aucun brevet n'a donc été déposé.

Dépôts de code open source:
\begin{itemize}
    \item Depot 1
    \item Depot 2
    \item Depot 3
\end{itemize}
\newpage
\begin{landscape}
\subsection{Bilan et suivi des personnels recrutés en CDD (hors stagiaires)}
\ANRinfo{Ce tableau dresse le bilan du projet en termes de recrutement de personnels non permanents sur CDD ou assimilé. Renseigner une ligne par personne embauchée sur le projet quand l’embauche a été financée partiellement ou en totalité par l’aide de l’ANR et quand la contribution au projet a été d’une durée au moins égale à 3 mois, tous contrats confondus, l’aide de l’ANR pouvant ne représenter qu’une partie de la rémunération de la personne sur la durée de sa participation au projet.
Les stagiaires bénéficiant d’une convention de stage avec un établissement d’enseignement ne doivent pas être mentionnés.\\}

\ANRinfo{Les données recueillies pourront faire l’objet d’une demande de mise à jour par l’ANR jusqu’à 5 ans après la fin du projet.}


\begin{table}[htbp]
\footnotesize
    \centering
    \begin{tabular}{|p{1.85cm}|p{0.65cm}|p{1.5cm}|p{1.15cm}|p{1.75cm}|p{1.25cm}|p{1.75cm}|p{1.3cm}|p{1.35cm}|p{1.35cm}|p{1.35cm}|p{1.25cm}|p{1.25cm}|p{1.25cm}|p{1.5cm}|p{1.5cm}|}
    \hline
Nom et prénom & Sexe& Adresse email & Date des dernières nouvelles
 & Dernier diplôme obtenu au moment du recrutement  & 
Lieu d'études (France, UE, hors UE)  & 
Expérience prof. Antérieure, y compris post-docs (ans)  & 
Partenaire ayant embauché la personne  & 
Poste dans le projet  & 
Durée  missions (mois)  & 
Date de fin de mission sur le projet & 
Devenir professionnel & 
Type d’employeur & 
Type d’emploi  & 
Lien au projet ANR & 
Valorisation expérience \\\hline

Zaiani Rim&F&rimazayani94 @gmail.com&date&Master 2 École Nationale Ingénieurs de Tunis&Hors UE&0&CESBIO&Doct.&18&&&&&&\\\hline
Giry-Fouquet Erwan&H&mail&date&Master 2 Data Mining Lyon2&France&0&CESBIO&Doct.&&&&&&&\\\hline
Baudoux Luc&H&mail&date&Master 2 IGAST&France&0&LASTIG&Doct.&36&&&&&&\\\hline
Chamfrault Maxime&H&mail&date&&France&0&LASTIG&Ingé&24&&&&&&\\\hline
Tagné Hermann&H&mail&date&&Hors UE&0&LASTIG&Ingé&&&&&&&\\

\hline
    \end{tabular}
    \label{tab:people}
\end{table}

\end{landscape}

