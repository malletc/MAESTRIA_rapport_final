\section{IMPACT DU PROJET}
\label{sec:impact}

\subsection{Indicateurs d’impact}
\subsubsection*{Nombre de publications et de communications}
Nous n'indiquons que les articles acceptés et publiés. Nous nous sommes focalisés sur des journaux et conférences internationales. Deux à trois articles de revue sont en plus à un stade d'écriture plus ou moins avancés mais non soumis.\\
Les détails sont fournis dans le tableau~\ref{tab:publis}.

\begin{table}[h!]
\small
    \centering
    \begin{tabular}{|p{3.2cm}|p{4.35cm}|c|c|}
    \hline
&  & Publications & Publications\\
 & &multipartenaires & monopartenaires \\
\hline

\multirow{3}{*}{International}&Revues à comité de lecture& 3&1\\\cline{2-4}
&Ouvrages ou chapitres d’ouvrage& -&-\\\cline{2-4}
&Communications (conférence)&2&3\\\hline
\multirow{3}{*}{National}&Revues à comité de lecture&-&-\\\cline{2-4}
&Ouvrages ou chapitres d’ouvrage&-&-\\\cline{2-4}
&Communications (conférence)&-&-\\\hline
\multirow{3}{*}{Actions de diffusion}&Articles vulgarisation&-&-\\\cline{2-4}
&Conférences vulgarisation&3&3\\\cline{2-4}
&Autres&-&-\\
\hline
    \end{tabular}

        \caption{Comptabilité des publications et communications.    \label{tab:publis}}

\end{table}


\subsubsection*{Autres valorisations scientifiques} 
Les détails sont fournis dans le tableau~\ref{tab:valo_other}.
\begin{table}[h!]
\small
    \centering
    \begin{tabular}{|p{6.5cm}|p{7.5cm}|}
    \hline
& Nombre, années et commentaires \\
&(valorisations avérées ou probables)\\\hline
Brevets internationaux &-\\\hline
Brevets nationaux &- \\\hline
Licences d’exploitation&- \\\hline
Créations d’entreprises ou essaimage&- \\\hline
Nouveaux projets collaboratifs & - \\\hline
Colloques scientifiques&- \\\hline
Autres & 11 dépôts de données code Open Source + Participation des deux partenaires aux développements de la librairie $\iota^2$ et à son \href{https://framagit.org/iota2-project/iota2/-/wikis/Project-Steering-Committee}{comité de pilotage}. \\
\hline
    \end{tabular}

    \caption{Comptabilité des autres valorisations.    \label{tab:valo_other}}
\end{table}


\subsection{Liste des publications et communications}

%% Nos publis ANR specifiquement dans ce bib
%% Merci de mettre le keyword ci-dessous pour que cela marche
\printbibliography[keyword={LASTIG-CESBIO-j},title={Journaux}]
\printbibliography[keyword={LASTIG-CESBIO-c},title={Conférences}]
%\printbibliography[keyword={LASTIG-CESBIO-a},title={Autres communications}]

\subsection{Liste des éléments de valorisation}
\label{subsec:valo}
Le projet s'est focalisé sur le renforcement des composantes méthodologiques de la chaîne logicielle libre $\iota^2$ (\href{https://framagit.org/iota2-project/iota2}{https://framagit.org/iota2-project/iota2}), la fourniture des codes de recherche plus amont et aval à la chaîne existante et à la mise à disponibtion des données et résultats généréés. Les codes développés ont suivi la licence de $\iota^2$ (GNU Affero General Public License v3.0). Aucun brevet n'a donc été déposé. La quantification exacte des retombées d'un tel projet, en particulier à court terme, est difficile à mener.

\paragraph{Dépôts de code open source:}
\begin{itemize}
    \item \href{https://src.koda.cnrs.fr/mmdc}{Multi Modal Data Cube (MMDC)}.
    \item Les développements nécessaires à MAESTRIA pour MMDC sont recensés en fin de Section~\ref{subsec:mmdc}.
    \item \mathieu{gitlab RMDA}. Accompagne la publication \cite{GIRYFOUQUET2021320}.
    \item \href{https://github.com/LBaudoux/Unet_LandCoverTranslator}{Land cover translation with an assymetrical-Unet}. Accompagne la publication \cite{Luc_RS}.
    \item \href{https://github.com/LBaudoux/MLULC}{Multi Land-use/Land-cover Translation Network}. Accompagne la publication \cite{luc_ijgis}.
\end{itemize}

\paragraph{Dépôts de données et résultats en open source:}
\begin{itemize}
    \item \href{https://zenodo.org/records/4459484}{Oso to Corine Land cover dataset}. Accompagne la publication \cite{Luc_RS}.
    \item \href{https://zenodo.org/records/5843595}{Multiple Land-use / Land-cover Dataset (MLULC)}. Accompagne la publication \cite{luc_ijgis}.
\end{itemize}

\newpage
\begin{landscape}
\subsection{Bilan et suivi des personnels recrutés en CDD (hors stagiaires)}

%\begin{sidewaystable}[htbp]
\begin{table}[htbp]
\tiny
    \centering
    \begin{tabular}{|m{1.5cm}|m{0.3cm}|m{1.5cm}|m{1.15cm}|m{1.25cm}|m{1.2cm}|m{0.95cm}|m{1.cm}|m{0.9cm}|m{0.9cm}|m{1.1cm}|m{1.cm}|m{1.cm}|m{0.85cm}|m{1.1cm}|m{1.cm}|}
    \hline
Nom et prénom & Sexe& Adresse email & Date des dernières nouvelles
 & Dernier diplôme obtenu au moment du recrutement  & 
Lieu d'études& 
Expérience prof. Antérieure, y compris post-docs (ans)  & 
Partenaire ayant embauché la personne  & 
Poste dans le projet  & 
Durée  missions (mois)  & 
Date de fin de mission sur le projet & 
Devenir professionnel & 
Type d’employeur & 
Type d’emploi  & 
Lien au projet ANR & 
Valorisation expérience \\\hline

Zaiani Rim&F&rimazayani94 @gmail.com&\jordi{??}&M2 École Nationale Ingénieurs de Tunis&Hors UE&0&CESBIO&Doct.&18&&&&&&\\\hline
Giry-Fouquet Erwan&H& erwan.giry.fouquet @gmail.com&21/10/22&M2 Data Mining Lyon2&France&0&CESBIO&Doct.&36& 05/11/22&  Data Scientist @ Liberty Rider &Start-up&CDI&Série temporelles&Oui\\\hline
Baudoux Luc&H&mail&date&Master 2 IGAST&France&0&LASTIG&Doct.&36&28/02/23&Data Scientist @ CCR&Public&CDI&Apprentissage Profond&Oui\\\hline
Chamfrault Maxime&H&maxime.chamfrault @gmail.com&01/05/23&Master Spécialisé SILAT AgroParisTech&France&15&LASTIG&IR&24&30/04/23&Chef de projet @ Syndicat des Eaux IDF&Public&CDD&Non&Non\\\hline
Tagné Hermann&H&tahertagui @yahoo.fr&31/10/23&M2 Physique de l'Environnement Yaoundé1&Hors UE&0&LASTIG&IR/Doct.&4&31/05/23&Fonctionnaire camerounais&Public&CDI&Traitement de données géospatiales&Oui\\

      \hline
    \end{tabular}
    \label{tab:people}
%\end{sidewaystable}
\end{table}
 \end{landscape}

