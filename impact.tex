\section{IMPACT DU PROJET}
\label{sec:impact}


\ANRinfo{Ce rapport rassemble des éléments nécessaires au bilan du projet et plus globalement permettant d’apprécier l’impact du programme à différents niveaux.}

\subsection{Indicateurs d’impact}
\subsubsection*{Nombre de publications et de communications}
Nous n'indiquons que les articles acceptés et publiés. Deux à trois articles de revue sont en plus à un stade d'écriture plus ou moins avancés mais non soumis.

\begin{table}[htbp]
\small
    \centering
    \begin{tabular}{|p{3.25cm}|p{3.25cm}|c|c|}
    \hline
&  & Publications & Publications\\
 & &multipartenaires & monopartenaires \\
\hline

\multirow{3}{*}{International}&Revues à comité de lecture& 2&1\\\cline{2-4}
&Ouvrages ou chapitres d’ouvrage& -&-\\\cline{2-4}
&Communications (conférence)&2&3\\\hline
\multirow{3}{*}{National}&Revues à comité de lecture&-&-\\\cline{2-4}
&Ouvrages ou chapitres d’ouvrage&-&-\\\cline{2-4}
&Communications (conférence)&-&-\\\hline
\multirow{3}{*}{Actions de diffusion}&Articles vulgarisation&-&-\\\cline{2-4}
&Conférences vulgarisation&1&1\\\cline{2-4}
&Autres&-&-\\
\hline
    \end{tabular}
    \label{tab:publis}
\end{table}


\subsubsection*{Autres valorisations scientifiques} 
\begin{table}[htbp]
\small
    \centering
    \begin{tabular}{|p{5.5cm}|p{6cm}|}
    \hline
& Nombre, années et commentaires \\
&(valorisations avérées ou probables)\\\hline
Brevets internationaux obtenus&-\\\hline
Brevet internationaux en cours d’obtention&- \\\hline
Brevets nationaux obtenus&- \\\hline
Brevet nationaux en cours d’obtention&- \\\hline
Licences d’exploitation (obtention / cession)&- \\\hline
Créations d’entreprises ou essaimage&- \\\hline
Nouveaux projets collaboratifs &\clement{??} \\\hline
Colloques scientifiques&- \\\hline
Autres (préciser)& XX dépôts de code Open Source + Participation des deux partenaires aux développements de la librairie $\iota^2$ et à son \href{https://framagit.org/iota2-project/iota2/-/wikis/Project-Steering-Committee}{comité de pilotage}. \\
\hline
    \end{tabular}
    \label{tab:valo_other}
\end{table}


\subsection{Liste des publications et communications}

%% Nos publis ANR specifiquement dans ce bib
%% Merci de mettre le keyword ci-dessous pour que cela marche
\printbibliography[keyword={LASTIG-CESBIO-j},title={Journaux}]
\printbibliography[keyword={LASTIG-CESBIO-c},title={Conférences}]
\printbibliography[keyword={LASTIG-CESBIO-a},title={Autres communications}]

\subsection{Liste des éléments de valorisation}
Le projet s'est focalisé sur le renforcement des composantes méthodologiques de la chaîne logicielle libre $\iota^2$ (\href{https://framagit.org/iota2-project/iota2}{https://framagit.org/iota2-project/iota2}), la fourniture des codes de recherche plus amont et aval à la chaîne existante et à la mise à disponibtion des données et résultats généréés. Les codes développés ont suivi la licence de $\iota^2$ (GNU Affero General Public License v3.0). Aucun brevet n'a donc été déposé. La quantification exacte des retombées d'un tel projet, en particulier à court terme, est difficile à mener.

Dépôts de code open source:
\begin{itemize}
    \item Depot 1
    \item Depot 2
    \item Depot 3
\end{itemize}

Dépôts de données et résultats en open source:
\begin{itemize}
    \item Depot 1
    \item Depot 2
    \item Depot 3
\end{itemize}

\clement{Listing à faire quand C.4 est finalisé}

\newpage
% \begin{landscape}
\subsection{Bilan et suivi des personnels recrutés en CDD (hors stagiaires)}
\ANRinfo{Ce tableau dresse le bilan du projet en termes de recrutement de personnels non permanents sur CDD ou assimilé. Renseigner une ligne par personne embauchée sur le projet quand l’embauche a été financée partiellement ou en totalité par l’aide de l’ANR et quand la contribution au projet a été d’une durée au moins égale à 3 mois, tous contrats confondus, l’aide de l’ANR pouvant ne représenter qu’une partie de la rémunération de la personne sur la durée de sa participation au projet.
Les stagiaires bénéficiant d’une convention de stage avec un établissement d’enseignement ne doivent pas être mentionnés.\\}

\ANRinfo{Les données recueillies pourront faire l’objet d’une demande de mise à jour par l’ANR jusqu’à 5 ans après la fin du projet.}


\begin{sidewaystable}[htbp]
\tiny
    \centering
    \begin{tabular}{|m{1.85cm}|m{0.35cm}|m{1.25cm}|m{1.15cm}|m{1.25cm}|m{1.25cm}|m{1.25cm}|m{1.cm}|m{1.cm}|m{1.cm}|m{1.cm}|m{1.cm}|m{1.cm}|m{1.cm}|m{1cm}|m{1cm}|}
    \hline
Nom et prénom & Sexe& Adresse email & Date des dernières nouvelles
 & Dernier diplôme obtenu au moment du recrutement  & 
Lieu d'études (France, UE, hors UE)  & 
Expérience prof. Antérieure, y compris post-docs (ans)  & 
Partenaire ayant embauché la personne  & 
Poste dans le projet  & 
Durée  missions (mois)  & 
Date de fin de mission sur le projet & 
Devenir professionnel & 
Type d’employeur & 
Type d’emploi  & 
Lien au projet ANR & 
Valorisation expérience \\\hline

Zaiani Rim&F&rimazayani94 @gmail.com&date&Master 2 École Nationale Ingénieurs de Tunis&Hors UE&0&CESBIO&Doct.&18&&&&&&\\\hline
Giry-Fouquet Erwan&H& erwan.giry.fouquet @gmail.com&10/21/22&Master 2 Data Mining Lyon2&France&0&CESBIO&Doct.&36& 05/11/22&  Data Scientist @ Liberty Rider &Start-up&CDI&Série temporelles&Oui\\\hline
Baudoux Luc&H&mail&date&Master 2 IGAST&France&0&LASTIG&Doct.&36&&&&&&\\\hline
Chamfrault Maxime&H&mail&date&&France&0&LASTIG&IR&24&&&&&&\\\hline
Tagné Hermann&H&mail&date&&Hors UE&0&LASTIG&IR&&&&&&&\\

      \hline
    \end{tabular}
    \label{tab:people}
\end{sidewaystable}

% \end{landscape}

