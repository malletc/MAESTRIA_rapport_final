\section{R\'ESUM\'E CONSOLID\'E PUBLIC}
\label{sec:resume}

\ANRinfo{Ce résumé est destiné à être diffusé auprès d’un large public pour promouvoir les résultats du projet, il ne fera donc pas mention de résultats confidentiels et utilisera un vocabulaire adapté mais n’excluant pas les termes techniques. Il en sera fourni une version française et une version en anglais. Il est nécessaire de respecter les instructions ci-dessous.}

\subsection{Résumé consolidé public en français}
Multi-modAl Earth obServaTion Image Analysis 

\ANRinfo{Titre 1 : situe l’objectif  général du projet et sa problématique (150 caractères max espaces compris)}


\ANRinfo{Paragraphe 1 : (environ 1200 caractères espaces compris)
Le paragraphe 1 précise les enjeux et objectifs du projet : indiquez le contexte, l’objectif général, les problèmes traités, les solutions recherchées, les perspectives et les retombées au niveau technique ou/et sociétal}

\subsubsection*{Titre 2 : précise les méthodes ou technologies utilisées (150 caractères max espaces compris)}

\ANRinfo{Le paragraphe 2 indique comment les résultats attendus sont obtenus grâce à certaines méthodes ou/et technologies. Les technologies utilisées ou/et les méthodes permettant de surmonter les verrous sont explicitées (il faut éviter le jargon scientifique, les acronymes ou les abréviations).}


\subsubsection*{Résultats majeurs du projet (environ 600 caractères espaces compris)}

\ANRinfo{Faits marquants diffusables en direction du grand public, expliciter les applications ou/et les usages rendus possibles, quelles sont les pistes de recherche ou/et de développement originales, éventuellement non prévues au départ.\\
Préciser aussi toute autre retombée= partenariats internationaux, nouveaux débouchés, nouveaux contrats, start-up, synergies de recherche, pôles de compétitivité, etc.}


\subsubsection*{Production scientifique et brevets depuis le début du projet}(environ 500 caractères espaces compris) \\
\ANRinfo{Ne pas mettre une simple liste mais faire quelques commentaires. Vous pouvez aussi indiquer les actions de normalisation}


\subsubsection*{Illustration}

\ANRinfo{Une illustration avec un schéma, graphique ou photo et une brève légende. L’illustration doit être clairement lisible à une taille d’environ 6cm de large et 5cm de hauteur. Prévoir une résolution suffisante pour l’impression. Envoyer seulement des illustrations dont vous détenez les droits.}


\subsubsection*{Informations factuelles}
\ANRinfo{Rédiger une phrase précisant le type de projet (recherche industrielle, recherche fondamentale, développement expérimental, exploratoire, innovation, etc.), le coordonnateur, les partenaires, la date de démarrage effectif, la durée du projet, l’aide ANR et le coût global du projet, par exemple « Le projet XXX est un projet de recherche fondamentale coordonné par xxx. Il associe aussi xxx, ainsi que des laboratoires xxx et xxx). Le projet a commencé en juin 2006 et a duré 36 mois. Il a bénéficié d’une aide ANR de xxx € pour un coût global de l’ordre de xxx  € »}

Le projet ANR MAESTRIA est un projet de recherche fondamentale coordonné par Clément Mallet (LASTIG). Il associe aussi le CESBIO. Le projet a commencé en octobre 2019 et a duré 54 mois. Il a bénéficié d’une aide ANR de 568$\:$k€ pour un coût global de l’ordre de \textcolor{red}{??}$\:$k€.


\subsection{Résumé consolidé public en anglais}

Transposition en anglais (deepL quoi\ldots)