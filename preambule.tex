\documentclass[12pt, oneside, paper=A4, DIV=15, BCOR=0mm, abstract=true, headings=small]{scrartcl}

\usepackage[french]{babel}
\usepackage[utf8]{inputenc}

\usepackage{colortbl}
%\usepackage{lmodern}
%Libertine
%\usepackage{libertinus}
%\usepackage{libertinust1math}

\usepackage[T1]{fontenc}
%\usepackage{fontspec}
%% \setmainfont[Ligatures=TeX]{Verdana}


\usepackage[headsepline, footsepline]{scrlayer-scrpage}
\usepackage{sectsty}
\usepackage{lastpage}
% Typographie

%\usepackage[auto]{microtype}
%\clubpenalty = 10000
%\widowpenalty = 10000
%\displaywidowpenalty = 10000

\usepackage{graphicx}
\usepackage{multirow, multicol, booktabs}
\usepackage{amsmath,amssymb,amsthm}
\usepackage{threeparttable}
\usepackage{longtable}
\usepackage{rotating}
\usepackage{ltablex}
\usepackage{subfig}
\captionsetup[subtable]{position=top}
\usepackage{pdfpages}
\usepackage{listings}
\usepackage{lscape}
\usepackage{filecontents}
\usepackage[natbib,style=numeric,defernumbers]{biblatex}
\addbibresource{biblio.bib}

\usepackage{url}
\usepackage{ulem}
\urlstyle{same}

% Fussnoten, auch für Tabellen
\usepackage{footnote}
\makesavenoteenv{tabular} 

%\renewcommand*{\labelnamepunct}{\addcolon\addspace}


\deffootnote{1.5em}{1em}{%
  \makebox[1.5em][l]{\thefootnotemark}%
}

\interfootnotelinepenalty=10000


\addtokomafont{caption}{\small}
\setkomafont{captionlabel}{\sffamily\bfseries}
\setkomafont{subject}{\Large\bfseries}
\setkomafont{author}{\normalfont}
\setkomafont{date}{\normalfont}
\setkomafont{publishers}{\normalfont}


\makeatletter
% \renewcommand{\l@section}{\@dottedtocline{1}{0em}{0em}}
% \renewcommand{\l@subsection}{\@dottedtocline{2}{4ex}{3.6em}}
% \g@addto@macro{\@afterheading}{\vspace{-0.25\baselineskip}}
\renewcommand{\l@section}{\@dottedtocline{1}{0ex}{1.8em}}
\renewcommand{\l@subsection}{\@dottedtocline{2}{1.8em}{2.7em}}
% \g@addto@macro{\@afterheading}{\vspace{-0.25\baselineskip}}
\makeatother


%\renewcaptionname{ngerman}{\figurename}{Abb.}
\renewcaptionname{french}{\tablename}{Tableau} 

% Tabellenumgebungen mit Schriftgröße 10 und 7
\newenvironment{tabular10}{%
  \fontsize{10}{12}\selectfont\tabular
}{%
  \endtabular
}

\definecolor{bleu}{rgb}{0.12, 0.29, 0.49}
\definecolor{gris}{rgb}{0.91, 0.91, 0.91}

%% \renewcommand\thesection{\Alph{section}}
%% \usepackage{titlesec}
%% \titleformat{\section}{\large\bfseries\textcolor{bleu}}{\thesection}{1em}{}

%% \titleformat{\subsection}{\bfseries\textcolor{bleu}}{\thesubsection}{1em}{}

%% \titleformat{\subsubsection}{\itshape\textcolor{bleu}}{\thesubsubsection}{1em}{}

\usepackage[
pdftitle={ANR MAESTRIA - avancement},
pdfsubject={},
pdfauthor={Clément Mallet},
pdfkeywords={Earth Observation, Image Analysis, Multi-Modal, Land-cover, Operational}]{hyperref}
%\usepackage[french]{cleveref}

\hypersetup{colorlinks=true,linkcolor=bleu}

\newcommand{\todo}[1]{{\color{red} TODO: {#1}}}
\newcommand{\ANRinfo}[1]{\textcolor{blue}{\textit{#1}}}
\newcommand{\clement}[1]{clement: \textcolor{green}{#1}}
\newcommand{\jordi}[1]{jordi: \textcolor{red}{#1}}
\renewcommand{\thesection}{\Alph{section}}



