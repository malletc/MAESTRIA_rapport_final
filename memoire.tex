\section{M\'EMOIRE SCIENTIFIQUE}
\label{sec:memoire}

\ANRinfo{Maximum 5 pages. On donne ci-dessous des indications sur le contenu possible du mémoire. Ce mémoire peut être accompagné de rapports annexes plus détaillés.\\}

\ANRinfo{Le mémoire scientifique couvre la totalité de la durée du projet. Il doit présenter une synthèse auto-suffisante rappelant les objectifs, le travail réalisé et les résultats obtenus mis en perspective avec les attentes initiales et l’état de l’art. C’est un document d’un format semblable à celui des articles scientifiques ou des monographies. Il doit refléter le caractère collectif de l’effort fait par les partenaires au cours du projet. Le coordinateur prépare ce rapport sur la base des contributions de tous les partenaires. Une version préliminaire en est soumise à l’ANR pour la revue de fin de projet.\\} 

\ANRinfo{Un mémoire scientifique signalé comme confidentiel ne sera pas diffusé. Justifier brièvement la raison de la confidentialité demandée. Les mémoires non confidentiels seront susceptibles d’être diffusés par l’ANR, notamment via les archives ouvertes http://hal.archives-ouvertes.fr.
}

\subsection*{Mémoire scientifique confidentiel :  non}

\subsection{Résumé du mémoire}

\subsection{Enjeux et problématique, état de l’art}
\ANRinfo{Présenter les enjeux initiaux du projet, la problématique formulée par le projet, et l’état de l’art sur lequel il s’appuie. Présenter leurs éventuelles évolutions pendant la durée du projet (les apports propres au projet sont présentés en C.4).}


\subsection{Approche scientifique et technique}

\subsection{Résultats obtenus}
\ANRinfo{Positionner les résultats par rapport aux livrables du projet et aux publications, brevets etc. Revisiter l’état de l’art et les enjeux à la fin du projet.}

Exemple de citation pour test biblio \cite{Luc_RS}

\subsection{Exploitation des résultats}


\subsection{Discussion} 
\ANRinfo{Discussion sur le degré de réalisation des objectifs initiaux, les verrous restant à franchir, les ruptures, les élargissements possibles, les perspectives ouvertes par le projet, l’impact scientifique, industriel ou sociétal des résultats. }


\subsection{Conclusions}


